%! Author = TiagoRG
%! GitHub = https://github.com/TiagoRG

\chapter{Introdução}
\label{ch:introducao}
{
%%%
% Conteúdo da introdução aqui
\section{Fórmulas} \label{sec:formulas}
    Para a resolução deste trabalho, foram consideradas as seguintes fórmulas, após certas deduções:
    \begin{itemize}
        \item $B_{sol} = \mu_0 (\frac{N}{l}) I_s$ (1)
        \item $\frac{\Delta C}{C} = \frac{\frac{\Delta N}{\Delta l}}{\frac{N}{l}} + \frac{\Delta m}{m}$ (2)
        \item $B = C_c V_H$ (3)
        \item $\vec{B}(x) = \frac{\mu_0 I R^2}{2(R^2 + (x - x_0)^2)^{3/2}}$ (4)
        \item $C_c = \frac{\mu_0 \frac{N}{l}}{m}$ (5)
        \item $\frac{N}{l} = \frac{B_{max}}{B_x}$ (6)
    \end{itemize}

    Para algumas destas fórmulas foi usada a constante $\mu_0$ que representa a permeabilidade magnética do vácuo, e é equivalente a $4\pi \times 10^{-7}~~Tm/A$.
%%%
\section{Objetivos} \label{sec:objetivos}
    O objetivo deste trabalho foi, na parte A, calibrar uma sonda de efeito de Hall, obtendo assim a sua constante de calibração ($C_c$), e calcular o seu respetivo erro.

    \par Na parte B, foi calculado o campo magnético ao longo do eixo de duas bobines na configuração de Helmholtz, utilizando uma sonda de Hall. \ Também foi estimado o número de espiras das bobines.
}
