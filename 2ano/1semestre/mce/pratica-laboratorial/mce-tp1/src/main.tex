%! Author = TiagoRG
%! GitHub = https://github.com/TiagoRG

\documentclass{report}
\usepackage[T1]{fontenc} % Fontes T1
\usepackage[utf8]{inputenc} % Input UTF8
\usepackage[backend=biber, style=ieee]{biblatex} % para usar bibliografia
\usepackage{csquotes}
\usepackage[portuguese]{babel} %Usar língua portuguesa
\usepackage{blindtext} % Gerar texto automaticamente
\usepackage[printonlyused]{acronym}
\usepackage{hyperref} % para autoref
\usepackage{graphicx}
\usepackage{indentfirst}
\usepackage{float}
\usepackage{geometry}

\geometry{
    paper=a4paper,
    margin=45pt,
    includefoot
}

\bibliography{bibliography}


\begin{document}
%%
% Definições
%! Author = TiagoRG
%! GitHub = https://github.com/TiagoRG

\newcommand{\titulo}{Mecânica e Campo Eletromagnético - Trabalho Prático 1}
\newcommand\data{DATA}
\newcommand\autores{Tiago Garcia, Rúben Gomes, Bruno Santos}
\newcommand\autorescontactos{(114184) tiago.rgarcia@ua.pt, (113435) rlcg@ua.pt, (113446) brunommsantos@ua.pt}
\newcommand\departamento{Dept. de Eletrónica, Telecomunicações e Informática}
\newcommand\empresa{Universidade de Aveiro}

%
%%%%%% CAPA %%%%%%
%
\begin{titlepage}

\begin{center}
%
\vspace*{50mm}
%
{\Huge \titulo}\\
%
\vspace{10mm}
%
{\Large \empresa}\\
%
\vspace{10mm}
%
{\LARGE \autores}\\
%
\vspace{30mm}
%
\begin{figure}[h]
\center
\includegraphics{images/ua}\label{fig:ua-title-logo}
\end{figure}
\end{center}
\end{titlepage}

%%  Página de Título %%
\title{%
{\Huge\textbf{\titulo}}\\
{\vspace{20mm}}
{\Large \departamento\\ \empresa}
}
%
\author{%
    \autores \\
    \autorescontactos
}
%
\date{\today}
%
\maketitle

%\pagenumbering{roman}

%%%%%% RESUMO %%%%%%
\begin{abstract}
    O principal objetivo deste trabalho é estudar o comportamento de uma esfera em diferentes tipos de movimentos. Para alcançar os objetivos pretendidos, foi necessário realizar 3 experiências, sendo estas o lançamento horizontal, lançamento com ângulo variável e por último lançamento contra um pêndulo balístico. As medições de comprimentos apresentam um erro de 1mm, medições de massas apresentam um erro de 0.1g e medições de ângulos apresentam um erro de 0.1º. A exatidão na primeira parte foi de 97.9\%.
\end{abstract}


\tableofcontents
%\listoftables     % descomentar se necessário
%\listoffigures    % descomentar se necessário


%%%%%%%%%%%%%%%%%%%%%%%%%%%%%%%
\clearpage
\pagenumbering{arabic}


%%%%%% INTRODUÇÃO %%%%%%
%! Author = TiagoRG
%! GitHub = https://github.com/TiagoRG

\chapter{Introdução}
\label{ch:introducao}
{
%%%
% Conteúdo da introdução aqui
\section{Fórmulas} \label{sec:formulas}
    Para a resolução deste trabalho, foram consideradas as seguintes fórmulas, após certas deduções:
    \begin{itemize}
        \item $B_{sol} = \mu_0 (\frac{N}{l}) I_s$ (1)
        \item $\frac{\Delta C}{C} = \frac{\frac{\Delta N}{\Delta l}}{\frac{N}{l}} + \frac{\Delta m}{m}$ (2)
        \item $B = C_c V_H$ (3)
        \item $\vec{B}(x) = \frac{\mu_0 I R^2}{2(R^2 + (x - x_0)^2)^{3/2}}$ (4)
        \item $C_c = \frac{\mu_0 \frac{N}{l}}{m}$ (5)
        \item $\frac{N}{l} = \frac{B_{max}}{B_x}$ (6)
    \end{itemize}

    Para algumas destas fórmulas foi usada a constante $\mu_0$ que representa a permeabilidade magnética do vácuo, e é equivalente a $4\pi \times 10^{-7}~~Tm/A$.
%%%
\section{Objetivos} \label{sec:objetivos}
    O objetivo deste trabalho foi, na parte A, calibrar uma sonda de efeito de Hall, obtendo assim a sua constante de calibração ($C_c$), e calcular o seu respetivo erro.

    \par Na parte B, foi calculado o campo magnético ao longo do eixo de duas bobines na configuração de Helmholtz, utilizando uma sonda de Hall. \ Também foi estimado o número de espiras das bobines.
}



%%%%%%%%%%%%%%%%%%%%%%%%%%%%%%%%

% Capítulos
%! Author = TiagoRG
%! GitHub = https://github.com/TiagoRG

\chapter{Detalhes Experimentais Relevantes}
\label{ch:detalhes-experimentais-relevantes}
{
%%%
% Conteúdo da introdução aqui

\section{Parte A}
\label{sec:detalhes-experimentais-relevantes-parte1}

\subsection{Material}
\label{subsec:detalhes-experimentais-relevantes-parte1-material}

\begin{enumerate}
    \item Lançador de projéteis
    \item Suporte para o lançador de projéteis
    \item Sensor de movimento
    \item Sensor de movimento
    \item Sistema de controlo dos sensores
    \item Fita-métrica
    \item Bola metálica
\end{enumerate}

\begin{figure}[h]
    \center
    \includegraphics[width=0.8\textwidth]{images/montagem-experimental-parte1}\label{fig:montagem-experimental-parte1}
\end{figure}

\subsection{Procedimento}
\label{subsec:detalhes-experimentais-relevantes-parte1-procedimento}

Antes de iniciar qualquer procedimento experimental é necessário certificar que o lançador de projéteis está devidamente montado e que o sistema de controlo dos sensores está ligado e a funcionar corretamente.

\begin{enumerate}
    \item Preparar a montagem experimental como ilustrado na figura \ref{fig:montagem-experimental-parte1};
    \item Medir a distância entre os sensores de movimento;
    \item Carregar o lançador de projéteis com a bola metálica na posição de tiro curto (SHORT RANGE);
    \item Colocar o sistema de controlo dos sensores na posição de TWO GATES e carregar em START/STOP;
    \item Disparar o projétil e registar o valor de tempo obtidos;
    \item Efetuar 3 disparos e registar as respetivas medições.
\end{enumerate}

\pagebreak

\section{Parte B}
\label{sec:detalhes-experimentais-relevantes-parte2}

\subsection{Material}
\label{subsec:detalhes-experimentais-relevantes-parte2-material}

\begin{enumerate}
    \item Lançador de projéteis
    \item Suporte para o lançador de projéteis
    \item Alvo
    \item Fita-métrica
    \item Bola metálica
\end{enumerate}

\begin{figure}[h]
    \center
    \includegraphics[width=0.8\textwidth]{images/montagem-experimental-parte2}\label{fig:montagem-experimental-parte2}
\end{figure}

\subsection{Procedimento}
\label{subsec:detalhes-experimentais-relevantes-parte2-procedimento}

Antes de efetuar os lançamentos, é necessário verificar rigorosamente o ângulo de lançamento e fixar devidamente o alvo de modo a evitar imprecisões relacionadas com o mesmo.

\begin{enumerate}
    \item Preparar a montagem experimental como ilustrado na figura \ref{fig:montagem-experimental-parte2};
    \item Colocar o alvo a uma distância tal que a esfera caia sobre a sua superfície;
    \item Carregar o lançador de projéteis com a bola na posição de tiro curto (SHORT RANGE);
    \item Medir a altura de lançamento do projétil;
    \item Disparar o projétil e registar o alcance obtido;
    \item Efetuar 3 disparos e registar as respetivas medições para cada valor de ângulo (sendo esses ângulos: 34º, 38º, 40º e 43º).
\end{enumerate}

\pagebreak

\section{Parte C}
\label{sec:detalhes-experimentais-relevantes-parte3}

\subsection{Material}
\label{subsec:detalhes-experimentais-relevantes-parte3-material}

\begin{enumerate}
    \item Suporte para o lançador de projéteis
    \item Lançador de projéteis
    \item Bola metálica
    \item Pêndulo balístico
    \item Balança
    \item Fita-métrica
\end{enumerate}

\begin{figure}[h]
    \center
    \includegraphics[width=0.8\textwidth]{images/montagem-experimental-parte3}\label{fig:montagem-experimental-parte3}
\end{figure}

\subsection{Procedimento}
\label{subsec:detalhes-experimentais-relevantes-parte3-procedimento}


\begin{enumerate}
    \item Preparar a montagem experimental como ilustrado na figura \ref{fig:montagem-experimental-parte3};
    \item Medir as massas do projétil e do pêndulo balístico;
    \item Medir o comprimento do pêndulo;
    \item Carregar o lançador de projéteis com a bola na posição de tiro curto (SHORT RANGE);
    \item Disparar o projétil e registar o ângulo máximo descrito pelo pêndulo;
    \item Efetuar 5 disparos e registar as respetivas medições.
\end{enumerate}

%%%
}

%! Author = TiagoRG
%! GitHub = https://github.com/TiagoRG

\chapter{Análise e Discussão}
\label{ch:analise-discussao}
{
%%%
% Conteúdo da Análise e Discussão aqui

\section{Parte A}
\label{sec:analise-discussao-parte1} 

\subsection{Resultados} \label{subsec:analise-discussao-parte1-resultados}
    Para o cálculo da constante de calibração, foi utilizada a montagem referida no enunciado, e registados os valores de $I_s$ e $V_H$.
    \par No gráfico abaixo é representada a reta de aproximação da função $V_h = f(I_s)$ , elaborada pelo software Excel com base nos dados obtidos experimentalmente.

\begin{figure}[h]
    \centering
    \includegraphics[scale=0.6]{images/grafico1-parte-a.png}
    \caption{Gráfico de representação linear de $I_s$ e $V_H$}
    \label{fig:grafico1-parte-a}
\end{figure}

    Na seguinte tabela, estão denotados os valores experimentais utilizados na elaboração do gráfico apresentado acima, e o valor calculado para a constante de calibração, $C_c$.

\begin{figure}[h]
    \centering
    \includegraphics[scale=0.6]{images/tabela1-parte-a.png}
    \caption{Tabela de resultados (parte A)}
    \label{fig:tabela1-parte-a}
\end{figure}

\subsection{Análise} \label{subsec:analise-discussao-parte1-analise}
    Quanto aos resultados obtidos, o valor obtido de $C_c$ foi calculado usando a equação (5), e este tem um erro associado que foi calculado utilizando a equação (2). \ O desvio associado foi de $\Delta C_c = 2.03 \times 10^{-2}$.

\subsection{Discussão}
    Após as medições e cálculos efetuados, foi obtido o gráfico \ref{fig:grafico1-parte-a} e a equação da reta de aproximação cujos valores divergem minimamente dos medidos. \ O declive desta equação foi utilizado para o cálculo de $C_c$. 
    \par Como $\frac{N}{l}$ e $\mu_0$ são valores constantes, a única variável no cálculo de $C_c$ é apenas o declive da reta da função $f(I_s)$, que está relacionado com os valores medidos, logo estes são a única influência no erro.
    \par Estes erros podem ser, por exemplo, o erro associado aos instrumentos de medição, ou pequenas variações na calibração da sonda.
    
\pagebreak

\section{Parte B}
\label{sec:analise-discussao-parte2}

\subsection{Resultados}
\label{subsec:analise-discussao-parte2-resultados}
    Na parte B, foram efetuadas medições em 3 partes, repartidas em 3 tabelas e gráficos. \ Estas representam a variação dos campos magnéticos gerados pelas bobines, consoante a posição da sonda de Hall.
    \par Na tabela 1, apenas a bobine imóvel tem corrente elétrica. \ Na tabela 2, apenas a bobine móvel tem corrente elétrica. \ Na tabela 3, as duas bobines estão ligadas em série, ambas com a mesma corrente elétrica. 
    \par Estas tabelas estão representadas na figura abaixo. 

    \begin{figure}[H]
        \centering
        \includegraphics[scale=0.6]{images/tabelas1-3-parte-b.png}
        \caption{Tabelas representativas dos campos magnéticos}
    \end{figure}

    Em baixo está representado os gráficos das respetivas tabelas em função à posição da sonda de Hall.

    \begin{figure}[H]
        \centering
        \includegraphics[scale=0.6]{images/graficos1-3-parte-b.png}
        \caption{Gráficos representativos das tabelas}
    \end{figure}

\subsection{Análise}
\label{subsec:analise-discussao-parte2-analise}
    O Princípio da Sobreposição do campo magnético consiste em que, numa configuração de Helmholtz, a soma do valor dos campos individualmente gerados numa dada posição da sonda de Hall será igual ao valor medido com as duas bobines ativas.

    \par Ao observar os resultados obtidos, a propriedade fundamental do Princípio da Sobreposição verifica-se de forma aproximada, ou seja, embora a soma dos campos individualmente medidos não seja exatamente igual ao valor medido aquando da medição simultânea dos dois campos, os valores são próximos.

    \par Nos gráficos está apresentado de forma mais clara este princípio. \ Com isto, podemos afirmar que ocorre sobreposição dos campos magnéticos.
    
    \par Para estimar o nº de espiras, foi utilizada a equação (4) para calcular o campo magnético no eixo de um anel de corrente. \ Após obter o resultado do campo magnético,  utiliza-se o $B_{max}$ obtido na tabela 3, usando a equação (6).


\subsection{Discussão}
\label{subsec:analise-discussao-parte2-discussao}
    Dado por completo os cálculos necessários, e a análise dos resultados obtidos, e a discrepância entre os valores teóricos e práticos, para a verificação do Princípio da Sobreposição do campo magnético. \ Estes desvios são originados por margens de erro, nos valores que este está dependente, como a constante de calibração e as medições da diferença de potencial numa dada posição.

    \par No cálculo do nº de espiras, o erro associado à estimativa deve-se a diversos fatores, como os erros associados ao cálculo do valor máximo obtido no campo magnético, e o cálculo de $B(x)$.

%%%
}


%%%%%% CONCLUSÕES %%%%%%
%! Author = TiagoRG
%! GitHub = https://github.com/TiagoRG

\chapter{Conclusões}
\label{ch:conclusoes}
{
%%%
% Conteúdo da conclusão aqui

Na primeira parte do trabalho, foi possível obter a constante de calibração da sonda de efeito de Hall, que foi aproximadamente $0.0309$. Não existiram grandes problemas na realização desta parte, pelo que foi possível chegar ao valor da constante de calibração sem existir nenhum desvio significativo.

Na segunda parte do trabalho, foi possível obter o campo magnético ao longo do eixo de duas bobines na configuração de Helmholtz, utilizando uma sonda de Hall. Também foi estimado o número de espiras das bobines, que foi aproximadamente $N = 80$ (cada bobine). Durante a experiência surgiram alguns problemas, nomeadamente erros na calibração da sonda, o que fez com que os valores obtidos não fossem imediatamente os esperados.

%%%
}



%%%%%% ACRÓNIMOS %%%%%%
%%! Author = TiagoRG
%! GitHub = https://github.com/TiagoRG

\chapter*{Acrónimos}
\begin{acronym}
    \acro{deti}[DETI]{Departamento de Eletrónica, Telecomunicações e Informática}
    \acro{leci}[LECI]{Licenciatura em Engenharia de Computadores e Informática}
    \acro{ua}[UA]{Universidade de Aveiro}
\end{acronym}



%%%%%%%%%%%%%%%%%%%%%%%%%%%%%%%%%
\printbibliography

\chapter*{Anexos}
\label{chap:anexos}



\end{document}
