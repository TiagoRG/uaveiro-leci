%! Author = tiagorg
%! Date = 31/01/2023
\documentclass[11pt]{report}

\usepackage{amsmath}
\usepackage[T1]{fontenc} % Fontes T1
\usepackage[utf8]{inputenc} % Input UTF8
\usepackage[backend=biber, style=ieee]{biblatex} % para usar bibliografia
\usepackage{csquotes}
\usepackage[portuguese]{babel} %Usar língua portuguesa
\usepackage{blindtext} % Gerar texto automaticamente
\usepackage{hyperref} % para autoref
\usepackage{graphicx}
\usepackage{indentfirst}
\usepackage[printonlyused]{acronym}
\usepackage{color}

\begin{document}
\def\titulo{Matemática Discreta}
\def\autores{Tiago Garcia}
\def\autorescontactos{tiago.rgarcia@ua.pt}
\def\empresa{Universidade de Aveiro}
\def\logotipo{ua.pdf}

%
\def\tema{Lógica de 1ª Ordem}
%

\begin{titlepage}
\begin{center}
\vspace*{50mm}
{\Huge\textbf{\titulo}}\\
\vspace{10mm}
{\Large \empresa}\\
\vspace{10mm}
{\LARGE \autores}\\
\vspace{30mm}
\begin{figure}[h]
    \center
    \includegraphics{ua}\label{fig:ua-title}
\end{figure}
\vspace{30mm}
\end{center}
\end{titlepage}
\title{
{\LARGE\textbf{\titulo} }\\\\
{\Large \aula\\ \empresa}
}
\author{
    \href{https://github.com/TiagoRG}{\autores} \\
    \href{mailto:tiago.rgarcia@ua.pt}{\autorescontactos}
}
\date{\today}
\maketitle
\pagenumbering{arabic}
\clearpage

    % Content

    \chapter*{Consequências Semânticas}
    \section*{Teorema}
    Uma fórmula $\Psi$ é consequência lógica (ou semântica) das fórmulas $\psi_1, \psi_2, \ldots, \psi_n$ se e só se $(\psi_1 \wedge \psi_2 \wedge \ldots \wedge \psi_n) \rightarrow \Psi$ é uma tautologia (fórmula válida).
    \subsection*{Notação}
    $ \psi_1, \ldots, \psi_n \models \Psi $\\
    $\Psi$ é consequência lógica (ou semântica) de $\psi_1, \ldots, \psi_n$\\
    \par $\psi_1, \ldots, \psi_n \vdash \Psi$ existe uma prova de $\Psi$ a partir de $\psi_1, \ldots, \psi_n$\\
    A prova recorre a regras de dedução designadas por regras de inferência, e a tautologias conhecidas.

    \section*{Teorema}
    $\psi_1, \ldots, \psi_n \models \Psi$\\
    ($\Psi$ é consequẽncia lógica de $\psi_1, \ldots, \psi_n$) se e só se o conjunto ${{\psi_1, \ldots, \psi_n, \neg\Psi}}$ é inconsistente, isto é, não existe uma interpretação para a qual todas as fórmulas do conjunto tomam valor 1.
    \par Para verificar se este conjunto de fórmulas é inconsistente usamos uma nova regra designada por resolução:\\\\
    $ \frac{\psi \rightarrow \theta~~~\Psi\vee\psi}{\theta\vee\psi} res $\\Indicam que aplicámos a regra/método da resolução.
    \subsection*{Casos particulares}
    \begin{enumerate}
        \item{Se $ \theta \equiv \bot $ obtemos\\
            $\frac{\Psi \rightarrow \bot~~~\Psi\vee\psi}{\bot\vee\psi}$\\
            simplificando como: $\bot\vee\psi\equiv\psi~~$ e $~~\Psi\rightarrow\bot\equiv\ned\Psi\vee\bot\equiv\ned\Psi$
            \par Para este caso particular a regra da resolução é:\\
            $\frac{\neg\Psi~~~\Psi\vee\psi}{\psi} res ~~ \rightarrow \neg\Psi, \Psi $ são lineares complementares.
        }
        \item {
            Se $ \theta\equiv\bot~~~e~~~\psi\equiv\bot $ (este é um caso particular do caso 1.)
            \par Se $\psi\equiv\bot$ então $\Psi\vee\psi\equiv\Psi\vee\bot\equiv\Psi$\\
            Substituindo no caso particular da regra de resolução obtida em 1. tem-se\\
            $ \frac{\neg\Psi~~~\Psi}{\bot} res $
        }
    \end{enumerate}

    \chapter*{Lógica Proposicional}
    \section*{Definição}
    \subsection*{Simbolos}
    Variáveis proposicionais: $p, q, \Psi, \psi, \ldots$\\
    Constantes: $\bot e \top$
    Conetivos lógicos: $\wedge, \vee, \rightarrow, \leftrightarrow, \neg, \equiv$
    \subsection*{Regras de construção}
    \begin{enumerate}
        \item{Se $\psi$ é uma fórmula proposicional então $\neg\neg\psi$ é uma fórmula proposicional.}
        \item{Se $\psi$ e $\theta$ são fórmulas proposicionais então $\psi\wedge\theta$ é uma fórmula proposicional.}
        \item{Se $\psi$ e $\theta$ são fórmulas proposicionais então $\psi\vee\theta$ é uma fórmula proposicional.}
        \item{Se $\psi$ e $\theta$ são fórmulas proposicionais então $\psi\rightarrow\theta$ é uma fórmula proposicional.}
        \item{Se $\psi$ e $\theta$ são fórmulas proposicionais então $\psi\leftrightarrow\theta$ é uma fórmula proposicional.}
    \end{enumerate}

    \section*{Dedução na lógica proposicional}
    \begin{itemize}
        \item {Verificar se uma fórmula é consequência lógica de um conjunto finito de fórmulas.\\
            $\psi_1, \ldots, \psi_n \models \Psi$
        }
        \item {Vimos que a consequência lógica é válida se e só se a implicação\\
            $\psi_1 \wedge \psi_2 \wedge \ldots \wedge \psi_n \rightarrow \Psi$ é uma tautologia.
        }
    \end{itemize}

    \subsection*{Para verificar se uma consequência lógica é válida:}
    \begin{enumerate}
        \item Verificar se a implicação associada é uma tautologia.
        \item Verificar se é possível obter (também são usados os termos deduzir, derivar, entre outros) $\Psi$ a partir de $\psi_1, \ldots, \psi_n$, recorrendo a regras de inferência e tautologias conhecidas (propriedades dos conetivos lógicos).\\ (através de uma sequência de deduções em que aplicamos as regras de inferências e tautologias), diz-se que existe uma prova de $\Psi$ a partir de $\psi_1, \ldots, \psi_n$ e escreve-se $\psi_1, \ldots, \psi_n \vdash \Psi$.
        \item Aplicar a regra de resolução - Método de resolução.
    \end{enumerate}
    \subsubsection*{Método de resolução}
    A consequência lógica $\psi_1, \ldots, \psi_n \models, \Psi$ é válida se e só se o conjunto de fórmulas {$\psi_1, \ldots, \psi_n, \neg\Psi$} é inconsistente, ou seja, este conjunto contém $\bot$ ou é possível deduzir $\bot$ a partir deste conjunto de fórmulas, isto é, existe uma prova de $\bot$ a partir de $\psi_1,\ldots,\psi_n,\neg\Psi$.

    \chapter*{Lógica de 1ª ordem}
    \section*{Definição} {
    Exemplo de uma fórmula da lógica proposicional:\\
    $(p \wedge q) \rightarrow r $\\
    Para traduzir frases do tipo:\\
    i) \color{red} todos \color{black} os gatos têm garras.\\
    ii) \color{red} alguns \color{black} alunos de MD têm 20.\\
    \par Passamos da lógica proposicional para a lógica de 1ª ordem (esta última engloba a outra).
    }
    \section*{Linguagem da lógica de 1ª ordem} {
    \subsection*{Alfabeto} {
    \begin{enumerate}
        \item Variáveis: x, y, z, \ldots;
        \item Conetivos lógicos da lógica proposicional: $\wedge, \vee, \rightarrow, \leftrightarrow, \neg, \equiv$;
        \item Constantes da lógica proposicional: $\bot e \top$;
        \item Os quantificadores $\forall~e~\exists$;
        \item O símbolo de igualdade: =;
        \item Símbolos de constantes;
        \item Símbolos de funções com aridade $n \in N$ (isto é, com $n$ argumentos);
        \item Símbolos de predicados.
    \end{enumerate}
    }
    \subsection*{Termo} {
    \begin{enumerate}
        \item Cada variável e cada símbolo de constante é um termo;
        \item { Se f é símbolo de função com aridade $n$ e $t_1, \ldots, t_n$ são termos então $f(t_1, \ldots, t_n)$ é um termo.\\\\
            Exemplo: {
            \begin{itemize}
                \item Variáveis: $x, y, z$;
                \item Constantes: $a = 1$, $b = $ Maria, $c = $ Gato tareco;
                \item Funções: pai(Maria), onde\\ Pai: $P\rightarrow P$, onde $P$ é o conjunto das pessoas.
                \item Predicado: $par(x)="x$ é par\("\), $D=N$\\ $par(2)=1,~~par(3)=0$, etc.
            \end{itemize}
            Como é que se constroem as fórmulas da lógica de 1.ª ordem?\\
            Definição (recursiva) de fórmula:
            \begin{itemize}
                \item $P(t_1, \ldots, t_n)$ é uma fórmla, considerando $P$ um simbolo de predicado e $t_1,\ldots,t_n$ termos.
                \item Se $\psi$ e $\Psi$ sao fórmulas então:\\ $\psi \wedge \Psi$, $\psi \vee \Psi$, $\psi \rightarrow \Psi$, $\psi \leftrightarrow \Psi$, $\neg\psi$, $\bot$ e $\top$ são fórmulas.
                \item Se $\psi$ é uma fórmula e $x$ é uma variável então $\forall x \psi$ e $\exists x \psi$ também são fórmulas.
                \item Se $t_1$ e $t_2$ são termos então $t_1 = t_2$ é uma fórmula.
            \end{itemize}
            }
        }
    \end{enumerate}
    }
    \subsection*{Átomo} {
    Na lógica proposicional, os átomos são as proposições atómicas (ex: $p =$ "chove", $q = $ "vou à aula de MD")\\
    \par Os átomos da lógica de 1ª ordem são:
    \begin{enumerate}
        \item $\bot, \top$
        \item $t_1=t_2$, com $t_1$ e $t_2$ termos
        \item $P(t_1,\ldots,t_n)$, onde $t_1,\ldots,t_n$ são termos e $P$ é um simbolo de predicado.
    \end{enumerate}
    \subsubsection*{Exemplo} {
    Consideremos os espaços vetoriais estudados na ALGA.\\
    O alfabeto inclui:
        \begin{itemize}
            \item O símbolo de constante o que representa o elemento nulo dos espaço vetorial
            \item Símbolos de funções
                \begin{enumerate}
                    \item Para cada $\alpha \in R$, o símbolo de funções\\ $\alpha \cdot \_$\\ que tem aridade 1 correspondente à multiplicação escalar.
                    \item O símbolo de função + com aridade 2, que corresponde à adição de elementos do espaço vetorial.
                \end{enumerate}
        \end{itemize}
    }
    \subsubsection*{Exemplos} {
    Converta as seguintes afirmações para linguagem simbólica da lógica de 1ª ordem:
    \begin{enumerate}
        \item{ \color{red}Todos \color{black} os gatos têm garras.\\
            \color{red} $\forall x$ \color{black} [$g(x) \rightarrow t(x)$]\\
            \color{red} Universo: $U$ = conjunto dos animais.
        }
        \item{ \color{red} Alguns \color{black} alunos de MD têm 20.\\
            \color{red} $\exists x$ \color{black} ($MD(x) \wedge V(x)$)\\
            MD(x) = "x é aluno de MD"\\
            V(x) = "x tem 20"
            \color{red} Universo: $U$ = alunos da UA em 22/23
        }
    \end{enumerate}
    }
    }
    }

    \section*{Folha 1}
    \subsection*{Exercício 2.}
    \subsubsection*{c)}
    \color{red} Todos \color{black} os insetos são mais leves do que \color{red} algum \color{black} mamífero.~~~~~\color{red} $\forall$ $\exists$\\
    \color{black} Predicados:\\
    $I(x)$ = ``x é um inseto``\\
    $L(y,z)$ = ``y é mais leve do que z``\\
    $M(w)$ = ``w é um mamífero``\\
    \par $\forall x \left(I(x) \rightarrow \exists y \left( M(y) \wedge L(x, y) \right) \right)$
    \par Obs: Alcance de cada quantificador:\\
    \begin{itemize}
        \item Ocorrência de x ligada: $I(x)$
        \item Alcance de $\forall x$: $\left(I(x) \rightarrow \exists y \left( M(y) \wedge L(x, y \right) \right)$
        \item Ocorrências de y ligadas: $M(y)$ e $L(x, y)$
        \item Alcance de $\exists y$: $\left( M(y) \wedge L(x, y \right)$
    \end{itemize}
    
    
    \chapter*{Fórmula fechada} {
    \section*{Definição} {
        Fórmula que não tem variáveis com occorrências livres.
        \subsection*{Exemplo} {
            $\forall x~\exists y~(P(x)~\rightarrow~R(x,y))$ é uma fórmula fechada.
            \par $\exists y~((\forall x~P(x))~\wedge~R(x,y))$, esta fórmula não é uma fórmula fechada.
        }
        \subsection*{Negação de fórmula com quantificadores} {
            \begin{enumerate}
                \item $\neg (\forall x~\psi)~\equiv~\exists x~\neg \psi$.
                \item $\neg (\exists x~\psi)~\equiv~\forall x~\neg \psi$.
            \end{enumerate}
            $\psi$ - parte da fórmula que está sob o quantificador.
        }
    }
    \section*{Introdução das fórmulas da lógica de 1ª ordem} {
        \subsection*{Definição} {
            \begin{itemize}
                \item Estrututa;
                \item Valoração,~~~V:$var~\rightarrow~D$, onde $D$ é o conjunto das variáveis.
            \end{itemize}
            O conceito de valoração pode ser entendido por forma a podermos considerar a valoração de um termo.\\
            $V(a) = a$, se $a$ é uma constante $V(f(t_1,\ldots,t_n)) = f^M(V(t_1),\ldots,V(t_n))$.
            \textbf{Obs:} Frequentemente denotamos o símbolo de função $f$ e a função correspondente na estrutura $f^M$, pela mesma letra.
        }
        \subsection*{Exemplo dos slides} {
            $V(M(A, x)) = M^M(V(A), V(x)) = M(A^M, 2) = M(1,2) = |1-2| = |-1| = 1$,~~~~~~$V(A) = A$ porque $A$ é uma constante.
        }
    }
    \section*{Interpretação de fórmulas} {
        \subsection*{Exemplo de interpretação de fórmulas (ver slides)} {
            \subsubsection*{i)} {
                Mostre que $R(x, A)$ não é válida na interpretação $(M,V)$\\
                \par Note-se que $\neg (M,V)\models R(x,A)$ se e só se $(M,V) \models \neg R(x,A)$ ($\neg R(x,A)$ é válida na interpretação $(M,V)$)\\
                \par $V(\neg R(x,A))\equiv\neg R(V(x),V(A))\equiv\neg R(2, A^M)\equiv\neg R(2, 1)\\\equiv\neg(2 < 1)\equiv\neg\bot\equiv\top $\\Logo, $\neg R(x, A)$ é valida na interpretação $(M,V)$, isto é, $(M,V)\models\neg R(x, A)$\\
                Isto é equivalente a afirmar que $R(x,A)$ não é válida nesta interpretação.
            }
        }
    }
    }
    \chapter*{Forma normal de Skolem} {
        \section*{Definição} {
            Uma fórmula $\phi$ é dita em forma normal de Skolem se $\phi$ é uma fórmula na forma normal conjuntiva e não contém nenhum quantificador universal.
        }
        \section*{Exemplo} {
            \subsection*{1)} {
                $\forall x~P(x, f(x))\wedge\neg R(x) $, onde $f$ é uma função e $R$ e $P$ são predicados.\\
            }
            \subsection*{2)} {
                $\forall x~\forall y~(P(x, f(x)) \wedge (\neg R(x)~\vee~P(x,y)))$
                \subsubsection*{Ideia} {
                    \begin{enumerate}
                        \item Convertemos $F$ numa fórmula $G$ que está na FNC prenex.\\ Note-se que $F \equiv G$
                        \item A partir de $G$ obtemos uma fórmula $H$ que está na forma normal de Skolem.
                    \end{enumerate}
                    \textbf{Para tal:}\\
                    \begin{itemize}
                        \item Se no início da fórmula temos um quantificador do tipo $\exists x$, substituimos todas as ocorrências de $x$ por um símbolo $a$ que represente uma constante e eliminamos o quantificador $\equiv x$.
                        \item Se na fórmula existe um quantificador existencial $\exists x_k$ com os quantificadores universais $\forall x_1~\forall x_2~\dots~\forall x_{k-1}$, à sua esquerda, substituimos todas as ocorrências de $x_k$ por um símbolo de função que ainda não esteja na fórmula, por exemplo $f$, que tem nos seus argumentos as variáveis $x_1, x_2, \dots, x_{k-1}$, isto é, $x_k$ é substituido por $f(x_1,\ldots,x_{k-1})$.\\\textbf{Atenção:} A fórmula $H$ que obtemos na forma normal de Skolen pode não ser (logicamente) equivalente à fórmula $G$ escrita na FNC prenex ou à fórmula $F$ original.
                    \end{itemize}
                }
            }
        }
    }


\end{document}
