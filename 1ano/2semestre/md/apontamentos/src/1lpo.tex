\part{Lógica de Primeira Ordem e Demonstração Automática}
\label{part:1lpo}

\chapter{Interpretação}
\label{chap:1-interpretacao}

\section{Proposição}
\label{sec:1-interpretacao-proposicao}

\subsection{Definição}
São proposições as afirmações que podem ser classificadas como verdadeiras ou falsas mas não ambas.

\subsection{Exemplos}
\begin{enumerate}
    \item O sol é uma estrela.
    \item Deus existe.
    \item D. Pedro I foi o primeiro imperador do Brasil.
\end{enumerate}

Afirmações com o seu valor lógico:
\begin{enumerate}
    \item Para todo o $n \in N$, $2n$ é múltiplo de $2$. $\rightarrow$ Proposição \textbf{Verdadeira}.
    \item Para todo o $n \in Z$, $2n \geq n$. $\rightarrow$ Afirmação \textbf{Ambígua}: \textbf{Verdadeira} para $n > 0$ e \textbf{Falsa} para $n \leq 0$.
    \item Para todo o $n \in N$, $3n \geq 4n$. $\rightarrow$ Proposição \textbf{False}.
\end{enumerate}

\subsection{Tipos de Proposições}

\begin{itemize}
    \item \textbf{Atómica}: Não pode ser decomposta em proposições mais simples.
    \item \textbf{Composta}: É formada a partir da combinação de proposições atómicas usando conectivos lógicos.
\end{itemize}

\section{Conectivos Lógicos}
\label{sec:1-interpretacao-conectivos}

\subsection{Negação}

\subsubsection{Símbolo}
O símbolo da negação é $\neg$.

\subsubsection{Tabela de Verdade}
\begin{table}[H]
    \begin{tabular}{|c|c|}
        \hline
        $p$ & $\neg p$ \\
        \hline
        1 & 0 \\
        0 & 1 \\
        \hline
    \end{tabular}
\end{table}

\subsection{Conjunção}

\subsubsection{Símbolo}
O símbolo da conjunção é $\land$.

\subsubsection{Tabela de Verdade}
\begin{table}[H]
    \begin{tabular}{|c|c|c|}
        \hline
        $p$ & $q$ & $p \land q$ \\
        \hline
        1 & 1 & 1 \\
        1 & 0 & 0 \\
        0 & 1 & 0 \\
        0 & 0 & 0 \\
        \hline
    \end{tabular}
\end{table}

\subsection{Disjunção}

\subsubsection{Símbolo}
O símbolo da conjunção é $\lor$.

\subsubsection{Tabela de Verdade}
\begin{table}[H]
    \begin{tabular}{|c|c|c|}
        \hline
        $p$ & $q$ & $p \lor q$ \\
        \hline
        1 & 1 & 1 \\
        1 & 0 & 1 \\
        0 & 1 & 1 \\
        0 & 0 & 0 \\
        \hline
    \end{tabular}
\end{table}

\subsection{Implicação}

\subsubsection{Símbolo}
O símbolo da conjunção é $\rightarrow$.

\subsubsection{Tabela de Verdade}
\begin{table}[H]
    \begin{tabular}{|c|c|c|}
        \hline
        $p$ & $q$ & $p \rightarrow q$ \\
        \hline
        1 & 1 & 1 \\
        1 & 0 & 1 \\
        0 & 1 & 0 \\
        0 & 0 & 1 \\
        \hline
    \end{tabular}
\end{table}

\subsection{Equivalência}

\subsubsection{Símbolo}
O símbolo da conjunção é $\leftrightarrow$.

\subsubsection{Tabela de Verdade}
\begin{table}[H]
    \begin{tabular}{|c|c|c|}
        \hline
        $p$ & $q$ & $p \leftrightarrow q$ \\
        \hline
        1 & 1 & 1 \\
        1 & 0 & 0 \\
        0 & 1 & 0 \\
        0 & 0 & 1 \\
        \hline
    \end{tabular}
\end{table}

\subsection{Exemplos}

\subsubsection{Exemplo 1}
Ou o José foi ao supermercado ou está sem ovos em casa.

\begin{itemize}
    \item $\phi$ = "O José foi ao supermercado"
    \item $\psi$ = "O José está sem ovos em casa"
\end{itemize}

\textbf{Resultado}: $\phi \lor \psi$

\subsubsection{Exemplo 2}
A Beatriz decidiu emigrar e não tenciona regressar.

\begin{itemize}
    \item $\phi$ = "A Beatriz decidiu emigrar"
    \item $\psi$ = "A Beatriz não tenciona regressar"
\end{itemize}

\textbf{Resultado}: $\phi \land \psi$

\subsubsection{Exemplo 3}

Ou o meu pai está em casa e a minha mãe não ou o meu pai não está em casa mas a minha mão está.

\begin{itemize}
    \item $\phi$ = "O meu pai está em casa"
    \item $\psi$ = "A minha mão não está em casa"
\end{itemize}

\textbf{Resultado}: $(\psi \land \neg \phi) \lor (\neg \psi \land \phi)$

\subsubsection{Exemplo 4}

Ficarei milionário se ganhar o euromilhões

\begin{itemize}
    \item $\phi$ = "Ficar milionário"
    \item $\psi$ = "Ganhar o euromilhões"
\end{itemize}

\textbf{Resultado}: $\psi \rightarrow \phi$

\section{Validade de Fórmulas}

\subsection{Tautologia}
\subsubsection{Definição}
Uma fórmula diz-se \textbf{Tautologia} quando tem valor lógico \textbf{1} para todas as suas interpretações.

Representa-se com $\top$.

\subsubsection{Exemplos}
\begin{enumerate}
    \item $\neg \psi \lor \psi$
    \item $(\psi \land \phi) \rightarrow \psi$
\end{enumerate}

\begin{table}[H]
    \begin{tabular}{|c|c|c|c|}
        \hline
        $p$ & $q$ & $p \land q$ & $(p \land q) \rightarrow q$ \\
        \hline
        0 & 0 & 0 & 1 \\
        0 & 1 & 0 & 1 \\
        1 & 0 & 0 & 1 \\
        1 & 1 & 1 & 1 \\
        \hline
    \end{tabular}
\end{table}

\subsection{Consistente}
\subsubsection{Definição}
Uma fórmula diz-se \textbf{Consistente} quando tem valor lógico \textbf{1} para alguma das suas interpretações.

\subsection{Inconsistente ou Contradição}
\subsubsection{Definição}
Uma fórmula diz-se \textbf{Inconsistente} ou \textbf{Contradição} quando tem valor lógico \textbf{0} para todas as suas interpretações.

Representa-se com $\bot$.

\subsubsection{Exemplo}
\begin{enumerate}
    \item $\neg \psi \land \psi$
\end{enumerate}

\section{Fórmulas Equivalentes}
\subsection{Definição}
As fórmulas $\phi$ e $\psi$ dizem-se equivalentes quando a fórmula $\phi \leftrightarrow \psi$ é uma tautologia.

\subsubsection{Demonstação}
\begin{table}[H]
    \begin{tabular}{|c|c|c||c|c|c|}
        \hline
        $p$ & $q$ & $p \rightarrow q$ & $\neg p$ & $\neg p \lor q$ & $(p \rightarrow q) \leftrightarrow (\neg p \lor q)$ \\
        \hline
        0 & 0 & 1 & 1 & 1 & 1 \\
        0 & 1 & 1 & 1 & 1 & 1 \\
        1 & 0 & 0 & 0 & 0 & 1 \\
        1 & 1 & 1 & 0 & 1 & 1 \\
        \hline
    \end{tabular}
\end{table}

\subsection{Exemplos}
\noindent Equivalências:
\begin{enumerate}
    \item $p \land q \equiv q \land p$
    \item $p \lor q \equiv q \lor p$
    \item $p \land (q \land r) \equiv (p \land q) \land r$
    \item $p \lor (q \lor r) \equiv (p \lor q) \lor r$
    \item $p \land p \equiv p$
    \item $p \lor p \equiv p$
    \item $p \land \top \equiv p$
    \item $p \lor \top \equiv \top$
    \item $p \land \bot \equiv \bot$
    \item $p \lor \bot \equiv p$
\end{enumerate}
Distributividade:
\begin{enumerate}
    \item $p \land (q \lor r) \equiv (p \land q) \lor (p \land r)$
    \item $p \lor (q \land r) \equiv (p \lor q) \land (p \lor r)$
\end{enumerate}
Leis de Morgan:
\begin{enumerate}
    \item $\neg (p \land q) \equiv \neg p \lor \neg q$
    \item $\neg (p \lor q) \equiv \neg p \land \neg q$
\end{enumerate}
Contraposição e dupla negação:
\begin{enumerate}
    \item $p \rightarrow q \equiv \neg q \rightarrow \neg p$
    \item $p \rightarrow q \equiv \neg p \lor q$
    \item $\neg \neg p \equiv p$
\end{enumerate}

\section{Formas Normais}
\label{sec:1-interpretacao-formas-normais}

\subsection{Literais}
\subsubsection{Definição}
Um literal é uma proposição atómica ou a negação de uma proposição atómica.

\subsubsection{Exemplos}
\begin{enumerate}
    \item $p$, $q$, $\neg r$ são literais.
    \item $\neg \neg p$, $p \rightarrow q$ não são literais.
\end{enumerate}

\subsection{Forma Normal Conjuntiva (FNC)}
Uma fórmula está na \textbf{Forma Normal Conjuntiva} se é uma conjunção de disjunção de literais.

\subsection{Forma Normal Disjuntiva (FND)}
Uma fórmula está na \textbf{Forma Normal Disjuntiva} se é uma disjunção de conjunções de literais.

\subsection{Exemplos}
\begin{enumerate}
    \item $p \land q \land \neg r$ está na FNC e na FND.
    \item $(p \lor \neg q) \land (q \lor r)$ está na FNC.
    \item $(p \land \neg q) \lor (q \land r)$ está na FND.
    \item $(p \land q) \lor (p \land \neg q) \lor (q \land r)$ não está na FNC nem na FND.
\end{enumerate}


